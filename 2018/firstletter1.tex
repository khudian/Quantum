

                 Dear colleagues and dear students,
    Many years  I had a dream to  give lectures  on Quantum Mechanics
in our School.  Two years ago I made a first attempt.
 Maybe some of my colleagues and  PhD students still remember it
(the short notes on the lectures which I gave in Autumn 2016 you may 
find on my  homepage in the subdirectory Teaching/Topics in 
Quantum Mechanics).
    This year I will recommence this course, but 
it will be much more detailed.  
The course will be oriented towards graduate students in
Mathematics, but of course anybody is welcome!
    There will be a two hour lecture
and a tutorial every week. This year the course will be formally assessed: 
there will be two assignments. (See the description of the course
in the file attached).
    I suggest the following  three slots for the lectures:
         A.  Tuesday            10-12
         B.   Wednesday         11-13
         C.   Thursday          11-13 
    May I ask the people interested 
to answer this e-mail  with the subject: "Quantum Mechanics".
It will be great if you specify your slot preference using  
the letters  A,B,C; e.g. $BCA$ (Wednesday > Thursday >Tuesday).
(People  undecided, may ignore this part.)
    Regarding the tutorial: the tutorial day will be fixed later
depending on the day chosen for the lecture.
Depending on the number of students the tutorial will last 
between one and two hours.
    Later I will create a mailing list for
the interested  people.
    
                   Thank you Hovik

   


\bye   



          Dear colleagues and dear students,
    There is a project that I have been thinking about for many years and
finally I decided to give it a try:
an experimental course ``Topics in Quantum Mechanics'.
    Over years I have heard from some PhD students that they would like to
learn basics of this subject. Some of them made attempts to attend the
corresponding course in the School of Physics, and these students told
me that it would be nice and more efficient if this course would be
delivered specially to mathematicians.
    I am planning a course oriented towards graduate students in
Mathematics. I plan to give about 8-10 lectures. Many things will depend
on the people who will come to listen the course.
    Below you can see a preliminary version of the programme of the course.
Sure it is a very preliminary version.
 	I need to know:
1) how many people approximately  will attend the course
2) your opinion about the programme.
    Those who are interested to attend the course, please respond to this
letter. I will be happy to receive any comments about the programme below.
 	After 21 September I will send the letter to fix the day
and time for the course.

                         Hovik
%%%%%%%%%%%%%%%%%%%%%%%%%%%%%%%%%%%%%%%%%%%%%%%%%%%%%%%%%%%%%%%%%
%%%%%%%%%%%%%%%%%%%%%%%%%%%%%%%%%%%%%%%%%%%%%%%%%%%%%%%%%%%%%%%%%
                   Topics in Quantum Mechanics
                   H.M.Khudaverdian
   Preliminary programme

 	1. Unitary space: complex linear space with Hermitian metric.
Self-adjoint operators in a unitary space. States and observables in
Quantum Mechanics. Measurement: commuting and non-commuting observables.
Cauchy-Bunyakovsky-Schwarz inequality and Heisenberg uncertainty principle.
 	2. Wave-function in Quantum Mechanics and action in Classical
Mechanics. Schroedinger equation. Coordinate and momentum
representations. Harmonic oscillator.
 	3. Rotation and angular momentum. Spin of a particle.
 	4. Perturbation theory: abrupt and adiabatic perturbations.
Adiabatic invariants in Quantum Mechanics and in Classical Mechanics.
 	5. Quasiclassical approximation in Quantum Mechanics and
Hamilton-Jacobi equation in Classical Mechanics. Fourier transform and
Legendre transform. Maslov index.
 	6. Elements of Quantum Logic. Modular lattice of questions
in Quantum Mechanics and distributive lattice of
questions in  Classical Mechanics.

%%%%%%%%%%%%%%%%%%%%%%%%%%%%%%%%%%%%%%%%%%%%%%%%%%%%%%%%%%%%%%%%%%



                                      Dr Hovhannes Khudaverdian
                                      The University of Manchester
                                         School of Mathematics
                                     e-mail: khudian@manchester.ac.uk
                                          tel. 0161-2008975



