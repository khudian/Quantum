\magnification=1200
\baselineskip=14pt
\def\vare {\varepsilon}
\def\A {{\bf A}}
\def\t {\tilde}
\def\a {\alpha}
\def\K {{\bf K}}
\def\N {{\bf N}}
\def\V {{\cal V}}
\def\s {{\sigma}}
\def\S {{\Sigma}}
\def\s {{\sigma}}
\def\p{\partial}
\def\vare{{\varepsilon}}
\def\Q {{\bf Q}}
\def\D {{\cal D}}
\def\G {{\Gamma}}
\def\C {{\bf C}}
\def\M {{\cal M}}
\def\Z {{\bf Z}}
\def\U  {{\cal U}}
\def\H {{\cal H}}
\def\R  {{\bf R}}
\def\S  {{\bf S}}
\def\E  {{\bf E}}
\def\l {\lambda}
\def\ll {{\bf l}}
\def\degree {{\bf {\rm degree}\,\,}}
\def \finish {${\,\,\vrule height1mm depth2mm width 8pt}$}
\def \m {\medskip}
\def\p {\partial}
\def\r {{\bf r}}
\def\pt {{\bf p}}
\def\v {{\bf v}}
\def\n {{\bf n}}
\def\t {{\bf t}}
\def\b {{\bf b}}
\def\c {{\bf c }}
\def\e{{\bf e}}
\def\ac {{\bf a}}
\def \X   {{\bf X}}
\def \Y   {{\bf Y}}
\def \x   {{\bf x}}
\def \y   {{\bf y}}
\def \z   {{\bf z}}
\def \G{{\cal G}}
\def\w {{\omega}}
\def \Tr  {{\rm Tr\,}}
\def\V {{\cal V}}
% I began this file in September  2018


\centerline {\bf Quantum mechanics. Problems 2.}


{\it 
          If  $\Psi \in\H$  is a state and $A$ is observable, then
the average value $\overline A =\left(\overline A\right)_\Psi$
of the observable (self-adjoint operator in $\cal H$)
                $$
      \left(\overline A\right)_\Psi=
                    {
            \langle\Psi,\hat A\Psi\rangle
                \over
       \langle\Psi,\Psi\rangle
                 }\,.
          \eqno (1)
                $$
In the case if we measure 
the averages of projection of the spin,
observables
 $s_x,s_y$ or $s_z$,  we may consider that the Hilbert 
vector space of
the states is $\C^2$:
            $$
\Psi=\pmatrix {a\cr b\cr}=a\uparrow+ b\downarrow\,,\quad a,b\in C\,,
           \eqno (2)
            $$
        
with the scalar product
$$
\langle\Psi,\Phi\rangle=
\left\langle \pmatrix {a\cr b\cr}\,,
\pmatrix {a'\cr b'\cr},
\right\rangle=a^*a'+b^*b'\,,
       $$
and observables are
           $$
s_x={1\over 2}\sigma_x={1\over 2}\pmatrix {0 & 1\cr 1 & 0\cr}\,,\quad
s_y={1\over 2}\sigma_y={1\over 2}\pmatrix {0 & -i\cr i & 0\cr}\,,\quad
s_z={1\over 2}\sigma_z={1\over 2}\pmatrix{1 & 0\cr 0 & -1\cr}\,,
        \eqno (3)
           $$ 
$\sigma_i$ are Pauli matrices,
  $[\sigma_k,\sigma_m]=i\vare_{kmn}\sigma_n$,
\footnote {$^*$}{$i\sigma_k$ is a natural basis 
in the Lie algebra $so(3)$}.

For example the average value of spin's $x$-component 
at the state $\Psi=\pmatrix {a\cr b\cr}$ is equal to
       $$
\overline {s_x}=      {
            \langle\Psi,s_x\Psi\rangle
                \over
       \langle\Psi,\Psi\rangle
                 }
      = {
            \langle\pmatrix {a\cr b\cr },{1\over 2}
               \pmatrix {0 & 1\cr 1 & 0\cr }
         \pmatrix {a\cr b\cr}
               \rangle
                \over
             \langle\pmatrix {a\cr b\cr },
         \pmatrix {a\cr b\cr}
               \rangle
                 }={a^*b+b^*a\over a^*a+b^*b}\,.
         \eqno (4)
       $$
}  


{\bf 1}i.   Find eigenfunctions of operators $s_x,s_y,s_z$.

\m

{\bf 2}.  Let $\Psi$ be a state such that
            $$
\Psi=c_{+}\uparrow_y
+c_{-}\downarrow_y\,,
            $$
where $\uparrow$ is eigenvector of  observable $s_y$ with eigenvalue
  $+{1\over 2}$, 
and $\downarrow$ is eigenvector of  observable $s_y$ with eigenvalue
  $-{1\over 2}$.  Show that
           $$
\overline {s_y}=
    {
{1\over 2}|c_+|^2
      -
{1\over 2}|c_-|^2
       \over 
|c_+|^2
      +
|c_-|^2
     }
           $$


\m


{\bf 3}.
Does there exist a state $\Psi$ such that for this state
$s_x$ and $s_y$ components of spin are exactly defined, i.e.
  the vector $\Psi\not=0$ is an eigenvector of two operators
$s_x$ and $s_y$ simultaneously?

   \m 


{\bf 4}
Consider a map
        $$
{\bf R}^2 \quad 
 {\buildrel f_1\over \rightarrow}
\quad {\bf C}\quad
{\buildrel f_2\over \rightarrow}\quad
         {\bf C}^2
\quad
 {\buildrel f\over \rightarrow}\quad
        S^2
           \eqno (4.1)
         $$
 such that $f_1$ is natural identification of the plane
$\R^2$ 
with the complex line  $\bf C$:
   $f_1(u,v)=u+iv$,    $f_2$ maps the complex number
   $z=u+iv$ into the spinor $\Psi=\pmatrix {z\cr 1\cr}$,
and the map $f_3$ maps, the spinor 
   $\Psi=\pmatrix {z\cr 1\cr}=\pmatrix {u+iv\cr 1\cr}$
into the point on $\S^2$ defined by the vector
               $$
{\bf s}=\left\{
   \overline {s_x}\,,
   \overline {s_y}\,,
   \overline {s_z}
        \right\}\,.
               $$
 Show that this map from $\R^2$ into the sphere is
the stereographic projection
\footnote{$^{**}$}
{{\bf Remark}  the map (4.1) in fact shows that
  $CP=S^2$ as homogeneous spaces:
   more presicely: group  $SU(2)$ acts on $\C P$,
and group $SO(3)$ on $S^2$, and the map (1.4)
intertwins the actions of these groups.
}.
\bye
