\magnification=1200
\baselineskip=14pt
\def\vare {\varepsilon}
\def\A {{\bf A}}
\def\t {\tilde}
\def\a {\alpha}
\def\K {{\bf K}}
\def\N {{\bf N}}
\def\V {{\cal V}}
\def\s {{\sigma}}
\def\S {{\Sigma}}
\def\s {{\sigma}}
\def\p{\partial}
\def\vare{{\varepsilon}}
\def\Q {{\bf Q}}
\def\D {{\cal D}}
\def\G {{\Gamma}}
\def\C {{\bf C}}
\def\M {{\cal M}}
\def\Z {{\bf Z}}
\def\U  {{\cal U}}
\def\H {{\cal H}}
\def\R  {{\bf R}}
\def\S  {{\bf S}}
\def\E  {{\bf E}}
\def\l {\lambda}
\def\ll {{\bf l}}
\def\degree {{\bf {\rm degree}\,\,}}
\def \finish {${\,\,\vrule height1mm depth2mm width 8pt}$}
\def \m {\medskip}
\def\p {\partial}
\def\r {{\bf r}}
\def\pt {{\bf p}}
\def\v {{\bf v}}
\def\n {{\bf n}}
\def\t {{\bf t}}
\def\b {{\bf b}}
\def\c {{\bf c }}
\def\e{{\bf e}}
\def\ac {{\bf a}}
\def \X   {{\bf X}}
\def \Y   {{\bf Y}}
\def \x   {{\bf x}}
\def \y   {{\bf y}}
\def \z   {{\bf z}}
\def \G{{\cal G}}
\def\w {{\omega}}
\def \Tr  {{\rm Tr\,}}
\def\V {{\cal V}}
% I began this file in September  2018


\centerline {\bf Quantum mechanics. Problems 3.}


{\it 
          If  $\Psi \in\H$  is a state and $A$ is observable, then
the average value $\overline A =\left(\overline A\right)_\Psi$
of the observable (self-adjoint operator in $\cal H$)
                $$
      \left(\overline A\right)_\Psi=
                    {
            \langle\Psi,\hat A\Psi\rangle
                \over
       \langle\Psi,\Psi\rangle
                 }
                $$
For example if $\H$  is realised as a space of funcions
 in $\E^3$, i.e.  
         $
    \Psi=\Psi(x,y,z)$  is a function such that
              $$
 \int_{\E^3} \Psi^*(x,y,z)\Psi(x,y,z)dxdydz <\infty\,.
               \eqno (1)
    $$ 
Then
the averages of coordinate $x$ is equal to 
                 $$
   <x>=
      \left(\overline x\right)_\Psi=
                    {
            \langle\Psi, x\Psi\rangle
                \over
       \langle\Psi,\Psi\rangle
                 }
           =
                    {
     \int_{-\infty}^\infty
        dx 
     \int_{-\infty}^\infty
        dy 
     \int_{-\infty}^\infty
        dz
      \left(
         x\Psi^*(x,y,z)\Psi(x,y,z)
      \right)
           \over
     \int_{-\infty}^\infty
        dx 
     \int_{-\infty}^\infty
        dy 
     \int_{-\infty}^\infty
        dz
      \left(
         \Psi^*(x,y,z)\Psi(x,y,z)
              \right)
            }\,,
        \eqno (2)
                 $$
and
the averages of the momentum $p_y$ is equal to
                 $$
   <p_y>=
      \overline
         { 
              \left(
          \hat {p_y}
              \right)
         }_\Psi=
                    {
            \langle\Psi, p_y\Psi\rangle
                \over
       \langle\Psi,\Psi\rangle
                 }
           =
                    {
     \int_{-\infty}^\infty
        dx 
     \int_{-\infty}^\infty
        dy 
     \int_{-\infty}^\infty
        dz
      \left(
         {\hbar\over i}
    {\p \Psi^*(x,y,z)\over \p y}
         \right)
           \Psi(x,y,z)
                 \over
     \int_{-\infty}^\infty
        dx 
     \int_{-\infty}^\infty
        dy 
     \int_{-\infty}^\infty
        dz
      \left(
         \Psi^*(x,y,z)\Psi(x,y,z)
              \right)
            }\,.
            \eqno (3)
                 $$


}

{\bf 1}.    
Consider the state
                  $$
                 \Psi=
        C 
      e^{-
          {
        \left(
    {\bf r}-{\bf r}_{_0}
       \right)^2
       \over 2a^2
           }
         }=
      C e^{-
           {
        \left(
    x-x_{_0}
       \right)^2
              +
        \left(
    y-y_{_0}
       \right)^2
        \left(
    z-z_{_0}
       \right)^2
           \over
            2a^2}
         }\,,
              $$
a) calculate the constant $C$ such that $\langle\Psi,\Psi\rangle=1$


b) Calculate averages of coordinates $x,y,z$

c) Calculate averages of momenta $p_x,p_y,p_z$ for this state. 

\m

{\bf 2}.
Let  $\Psi$ be an arbitrary state which is described by the real
function in the space $\E^3$:  
        $$
 \Psi^*=\Psi
        $$
Show that averages of momenta vanish for this state.

Explain why the condition (1) is important.

\m


{\bf 3}  Consider 
       $\Psi=e^{-x^2-y^2}z$.
Why you cannot  you use formula (3) to evaluate  the average of the 
momentum $p_z$?

\m


{\bf 4}.
For the state
        $$
\Psi(x,y,z)=C e^{-
           {
        \left(
    \r-\r_{_0}
       \right)^2
           \over
            2a^2}
         +{i{\bf p}_{_0}\cdot \r\over \hbar}}
= C e^{-
           {
        \left(
    x-x_{_0}
       \right)^2
              +
        \left(
    y-y_{_0}
       \right)^2
        \left(
    z-z_{_0}
       \right)^2
           \over
            2a^2}
         +{
         ip_{_{0x}}x+
         ip_{_{0y}}x+
         ip_{_{0z}}z
          \over \hbar}}
        $$
calculate
the averages,  
$\overline x$,
$\overline y$, 
$ \overline z$,
$ \overline p_x$,
$ \overline p_y$,
$ \overline p_z$,
and the dispersions 
$\overline {\Delta x^2}$
$\overline {\Delta y^2}$
$\overline {\Delta z^2}$
$\overline {\Delta p_x^2}$
$\overline {\Delta p_y^2}$
$\overline {\Delta p_z^2}$.

\bye
