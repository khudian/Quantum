\magnification=1200
\baselineskip=14pt
\def\vare {\varepsilon}
\def\A {{\bf A}}
\def\t {\tilde}
\def\a {\alpha}
\def\K {{\bf K}}
\def\N {{\bf N}}
\def\V {{\cal V}}
\def\s {{\sigma}}
\def\S {{\Sigma}}
\def\s {{\sigma}}
\def\p{\partial}
\def\vare{{\varepsilon}}
\def\Q {{\bf Q}}
\def\D {{\cal D}}
\def\G {{\Gamma}}
\def\C {{\bf C}}
\def\M {{\cal M}}
\def\Z {{\bf Z}}
\def\U  {{\cal U}}
\def\H {{\cal H}}
\def\R  {{\bf R}}
\def\S  {{\bf S}}
\def\E  {{\bf E}}
\def\l {\lambda}
\def\ll {{\bf l}}
\def\degree {{\bf {\rm degree}\,\,}}
\def \finish {${\,\,\vrule height1mm depth2mm width 8pt}$}
\def \m {\medskip}
\def\p {\partial}
\def\r {{\bf r}}
\def\pt {{\bf p}}
\def\v {{\bf v}}
\def\n {{\bf n}}
\def\t {{\bf t}}
\def\b {{\bf b}}
\def\c {{\bf c }}
\def\e{{\bf e}}
\def\ac {{\bf a}}
\def \X   {{\bf X}}
\def \Y   {{\bf Y}}
\def \x   {{\bf x}}
\def \y   {{\bf y}}
\def \z   {{\bf z}}
\def \G{{\cal G}}
\def\w {{\omega}}
\def \Tr  {{\rm Tr\,}}
\def\V {{\cal V}}
% I began this file in Novmber  2018


\centerline {\bf Quantum mechanics. Problems 4.}

  \centerline {Heisenberg uncertainty principle}

{\it One of the formulation of 
 the Heisenberg incertainty principle is the following:
  Let  $\hat A,\hat B$  be two observables,
and $\hat C$ be an operator such that
     $$
  \hat C=i[\hat A,\hat B]\,. 
\eqno (1)
     $$
Then  for any given state $\Psi\not=0$
the product of dispersions\footnote{$^*$}
{Dispersion of operator $\hat K$ on an arbitrary state $\Phi$
is equal to  
        $$
 \sqrt{\Delta K^2}=\sqrt{<\Phi, (\hat K-k) \Phi>}\,,
           $$
where $k=<\Psi,\hat K\Psi>$ is the average of the operator $\hat K$.
(We suppose that $<\Psi,\Psi>=1$.)
}  of observables $\hat A$ and $\hat B$
is bigger tor equal than the half of the dispesion
of $\hat C$.
}
\m

{\bf 1}.
a)  is the   operator $C'=-iC=[A,B]$ in (1) an observable?

b)  explain why the   operator $C=i[A,B]$  is an observable.

\m


{\bf 2}  Suppose that for the state $\Psi\not=0$,
    $$
 <A>_{\rm average}=a\,,\quad
 <B>_{\rm average}=b\,.
        $$
Show that  averages of  operators $A'=A-a$ and $B'=B-b$ on the state 
$\Psi$ vanish.


            
\m


{\bf 3} Show that the 
dispersions of operator $A$ and operator $B$
on the state $\Psi$ coincide respectively with the
dispersions of the operator $A'$ and operator $B'$
on this state

\m

{\bf 4}     Using this result prove the Heisenberg uncertainty 
principle.

\m

{\bf 5} ({\tt Uncertainty principle in music}) 
Consider the signal
       $$
   A(t)=\cases 
          {
    0 \,\, \rm {for}\,\,  t<0\cr
    A_0\sin w_0 t\,\, {\rm for}\,\,  0<t<T\cr
    0\,\, {\rm for} t>T
           }
           $$     

 


 
    Calculate   the dispersion $\sqrt {\Delta w^2}$ of frequency.

   Explain why $\sqrt {\Delta w^2}\to \infty$ if $T\to 0$. 



\m


{\bf 6}  

  


Consider  a state
     $$
\Psi_1(x,y,z)=e^{(x-x_0)^2+i{p_{_0}y\over \hbar} }
     $$
Is it possible to measure simultaneously $x$ and $p_y$
for this state?

   Does this contradicts to Heisenberg uncertainty principle?



\bye





{\bf 6}  Let $\Psi=a\uparrow+b\downarrow$  be an arbitrary non-zero
  spinor.

 a) Show that $\Psi$ cannot be the common 
 eigenvector of Pauli matrices  $\sigma_x$
and $\sigma_y$.


b) Show that there exist parameters $a,b$ and $c$  such that
  the matrix $a\sigma_x+b\sigma_y+c\sigma_z$
is not vanished, and   
$\Psi$ is the eigenvector of this non-zero matrix.

Explain the physical meaning of this result.


\bye
