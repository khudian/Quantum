\magnification=1200
\baselineskip=14pt
\def\vare {\varepsilon}
\def\A {{\bf A}}
\def\t {\tilde}
\def\a {\alpha}
\def\K {{\bf K}}
\def\N {{\bf N}}
\def\V {{\cal V}}
\def\s {{\sigma}}
\def\S {{\Sigma}}
\def\s {{\sigma}}
\def\p{\partial}
\def\vare{{\varepsilon}}
\def\Q {{\bf Q}}
\def\D {{\cal D}}
\def\G {{\Gamma}}
\def\C {{\bf C}}
\def\M {{\cal M}}
\def\Z {{\bf Z}}
\def\U  {{\cal U}}
\def\H {{\cal H}}
\def\R  {{\bf R}}
\def\S  {{\bf S}}
\def\E  {{\bf E}}
\def\l {\lambda}
\def\ll {{\bf l}}
\def\degree {{\bf {\rm degree}\,\,}}
\def \finish {${\,\,\vrule height1mm depth2mm width 8pt}$}
\def \m {\medskip}
\def\p {\partial}
\def\r {{\bf r}}
\def\pt {{\bf p}}
\def\v {{\bf v}}
\def\n {{\bf n}}
\def\t {{\bf t}}
\def\b {{\bf b}}
\def\c {{\bf c }}
\def\e{{\bf e}}
\def\ac {{\bf a}}
\def \X   {{\bf X}}
\def \Y   {{\bf Y}}
\def \x   {{\bf x}}
\def \y   {{\bf y}}
\def \z   {{\bf z}}
\def \G{{\cal G}}
\def\w {{\omega}}
\def \Tr  {{\rm Tr\,}}
\def\V {{\cal V}}
% I began this file in Novmber  2018


\centerline {\bf Quantum mechanics. Problems 5.}

\m


{\bf 1}   

\centerline {\tt Coordinate and Momentum representation}
           $$
\Psi(\r)={1\over (2\pi \hbar)^{3\over 2}}
          \int \Phi({\bf p})e^{{i\over \hbar}{\bf p}\r}d^3r\,,
\qquad
\Phi({\bf p})={1\over (2\pi \hbar)^{3\over 2}}
          \int \Psi(\r)e^{-{i\over \hbar}{\bf p}\r}d^3p\,.
           \eqno (1)
           $$

If the vector ${\bf x}$  defines a state of quantum mechanical
system, then $\x=\Psi(r)$  is the representation of this state
in coordinates, and
   $|\Psi(\r_0)|^2$ says  about a probability to find a particle
in the vicinity of the point $\r_0$; respectively
If $\x=\Phi(p)$ is the representation of this state
in momenta, and 
   $|\Phi({\bf p}_0)|^2$ `says'  about a probability that momentum of the
particle
is the vicinity of the $\bf p_0$.


{a)}    Check that formulae (1) are well defined, in particularly check that
double Fourier transform in equation (1) is the identity operation.

  {\it Hint:  you may use the formula that
                    $\int_{-\infty}^{\infty} e^{ikx}dk=2\pi \delta(x)$.}

 
             $$
\Psi\rightarrow \Phi\rightarrow \Psi
             $$




{b)} Find momentum representation  for wave function
$\Psi(x)=\delta(x-x_0)+\delta (x-x_1)$.

   Explain why average momentum of this state vanishes.




c) Find momentum representatin for the wave function
 $\Psi (x)=\cases {
          1\quad \hbox {for $0<x<a$}\cr 
          0 \quad\hbox {for $x\not\in (0,a)$}\cr 
                      }$




\m


{\bf 2}  

a) Calculate 
         $$
e^{{i\over \hbar}\hat p_x}   e^{-(x-x_0)^2}\,.
          $$


b) You can see the following exercise in Quantum Mechanics:

  {\tt Show that}
            $$
e^{{i\over \hbar}\hat p_x}   \Psi(x)=\Psi(x+1)
                 $$ 



Is this statement correct  ({\tt for an arbitrary smooth function})?

\m




{\bf 3}  

a) Let  $A$ be an antisymmetric  matrix.  Show that
   $e^{A}$ is an orthogonal  matrix.

\m


b) Calculate the mtrix $e^{t\pmatrix {0 & 1\cr -1 & 0\cr}}$.


\m

c) Let  $B$ be hermitian matrix.  Show that
   $e^{iB}$ is unitary matrix.

\m

d) Calculate the matrix $e^{i\varphi\pmatrix {0 & 1\cr -1 & 0\cr}}$.



\m


{\bf 4}   At the moment $t=0$, the wave function of the free particle
is
       $$
\Psi(x)=e^{-(x-x_0)^2\over 2a^2}
       $$
How look this wave function at the moment 
$t=t_0$?

\m

{\bf 5}

a) Find stationary states of the particle in the infinite potential
well of the width $a$.


b)   Find the momeuntum representation of the state
with minimum energy.

c) Calculate the average momentum of the arbitrary stationary state.

\bye
