\magnification=1200
\baselineskip=14pt
\def\vare {\varepsilon}
\def\A {{\bf A}}
\def\t {\tilde}
\def\a {\alpha}
\def\K {{\bf K}}
\def\N {{\bf N}}
\def\V {{\cal V}}
\def\s {{\sigma}}
\def\S {{\Sigma}}
\def\s {{\sigma}}
\def\p{\partial}
\def\vare{{\varepsilon}}
\def\Q {{\bf Q}}
\def\D {{\cal D}}
\def\G {{\Gamma}}
\def\C {{\bf C}}
\def\M {{\cal M}}
\def\Z {{\bf Z}}
\def\U  {{\cal U}}
\def\H {{\cal H}}
\def\R  {{\bf R}}
\def\S  {{\bf S}}
\def\E  {{\bf E}}
\def\l {\lambda}
\def\ll {{\bf l}}
\def\degree {{\bf {\rm degree}\,\,}}
\def \finish {${\,\,\vrule height1mm depth2mm width 8pt}$}
\def \m {\medskip}
\def\p {\partial}
\def\r {{\bf r}}
\def\pt {{\bf p}}
\def\v {{\bf v}}
\def\n {{\bf n}}
\def\t {{\bf t}}
\def\b {{\bf b}}
\def\c {{\bf c }}
\def\e{{\bf e}}
\def\ac {{\bf a}}
\def \X   {{\bf X}}
\def \Y   {{\bf Y}}
\def \x   {{\bf x}}
\def \y   {{\bf y}}
\def \z   {{\bf z}}
\def \G{{\cal G}}
\def\w {{\omega}}
\def \Tr  {{\rm Tr\,}}
\def\V {{\cal V}}
\def\H {{\cal H}}
% I began this file in 25 November  2018


\centerline {\bf Quantum mechanics. Problems 7.}

\centerline {\tt Secondary Quantisation}

{\it 
We consider the Hamiltonian
$$
\hat\H=
\sum_i\left(
  {\hat p_i^2\over 2m}+{mw_i^2\hat q_i^2\over 2}
\right)=
\sum_i E_i\left(
a_i^+a_i+{1\over 2}
\right)\,,\quad
\sum_i\hbar w_i\left(
a_i^+a_i+{1\over 2}
\right)\,,\quad
    E_i=\hbar w_i
    \eqno (1)
 $$
This Hamiltonian simultaneousy describes
the following two pictures

{\bf I}  free (non-interacting) harmonic osicillators,
every oscillator with frequency $w_i$;


{\bf II} free non-relativistic {\it identical bosonic}
 particles.  Each of these particles
is described by the Hamiltonian
         $$
   H_{\rm classical}={p^2\over 2m}+U(q)\,,\qquad
   H_{\rm quantum}={{\hat p}^2\over 2m}+U({\hat q})\,,\qquad
    \eqno (2)
         $$
In the second picture  classical equations of motion
Equations of motion are
         $
\cases {
     \dot p=-{\p H\over \p q}=-{\p U\over \p q}\cr
     \dot q={\p H\over \p q}=p\cr
         }  
         $
of one particle  after qunatisation
become the following 
        $$
i\hbar {\p \Psi(q,t)\over \p t}={\hat H}\Psi\,,
\qquad \Psi (q,t)=\sum_i c_i(t)\varphi_i(x)\,,\quad
{\rm where}\,\,
 \hat H\varphi_n=E_n\varphi_n\,.
    \eqno (3)
        $$
}

\m


{\bf 1}   Show that $\{c_n(t)\}$  in equation (3) 
obey the differential equations
         ${dc_n(t)\over dt}=E_nc_n(t)$.

\m


{\bf 2} Show that these equations are  equations of 
 motion of classical Hamiltonian 
      $$
\H=\H(c,c^*)=\sum_i E_ic_i^*c_i\,,\quad \hbox {with Poisson bracket}
   \{c_i,c_j\}=
   \{c_i^*,c_j^*\}=0\,,\quad
   \{c_i,c_j^*\}={1\over i\hbar}\delta_{ij}\,.
      $$

\m



{\bf 3}  Show that the Hamiltonian (1)  describing 
quantum osillators will be
the quantum Hamiltonian describing free particles.


\m



{\bf 4}   Consider a state $\Psi$
such that 


i) the first oscillator is in the first state,  $n_1=1$,
i.e. its energy is equal to 
   $$
   \hbar w_1\left(n_1+{1\over 2}\right)=
{3\over 2}\hbar w_1\,,
      $$

ii) the second oscillator is in the second  state,  $n_2=2$,
i.e. its energy is equal to 
   $$
   \hbar w_1\left(n_2+{1\over 2}\right)=
{5\over 2}\hbar w_1\,,
      $$


all other oscillators are in the ground state:
   $n_3=n_4=\dots=0$, i.e. their energies are equal to
     respectively to ${1\over 2}\hbar w_i$.



a) write down the wave function 
 $\Psi=\Psi(x_1,x_2,x_3,\dots)$  of 
these osillators in coordinate representation


b) write down the wave  function $\hat a_2\Psi$


{\it the wave function $\Psi=|12>$  corresponds in the second picture
 to the wave function of
3 particles: one particle  at the energy $E_1$ and two particles 
at the energy $E_2$}


c) write down this wave-function in terms of wave functions
$\{\varphi_i(x)\}$ ( eigenfunctions of one particle: 
  $\hat H\varphi_n=E_n\varphi_n$.)
\bye
