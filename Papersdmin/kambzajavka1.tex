From Mark.Kambites@manchester.ac.uk Mon Dec 11 13:40:49 2017
Date: Mon, 11 Dec 2017 13:41:08 +0000
From: Mark Kambites <Mark.Kambites@manchester.ac.uk>
To: Hovhannes Khudaverdian <khudian@manchester.ac.uk>
Subject: Re: The course of Quantum Mechanics for PhD students

Dear Hovik,

I sent it to Matthias (who needs to approve it to count towards the taught 
course requirement for PhD students) and he asked for the "up to 11 
tutorial sessions" to be made more specific (I would personally suggest 
just "11 tutorial session"!) and for a bit more detail about the 
assignments:

> Can we change the "up to 11 tutorial sessions" to something more 
> specific, and provide some more detail about the nature of the assignments?

Could you please edit slightly to address these and let me have a revised 
PDF?

Thanks! Mark



On Mon, 11 Dec 2017, Hovhannes Khudaverdian wrote:

 > 
 > 
 >   Dear Mark
 >   Sorry I have forgotten to write it.
 >   Sure  the  first semester
 > (I have two lecture courses in the second semester)
 > 
 > 
 >      Thank you very much
 > 
 >              Hovik
 > 
 > On Mon, 11 Dec 2017, Mark Kambites wrote:
 > 
 > > Dear Hovik,
 > >
 > > Thanks for this. Perhaps you told me this and I forgot, but do you have a
 > > preference for which semester it runs in?
 > >
 > > Best wishes, Mark
 > >
 > >
 > >
 > > On Tue, 5 Dec 2017, Hovhannes Khudaverdian wrote:
 > >
 > > >
 > > >                  Dear Mark,
 > > > I send you attached to this  letter the pdf file of the programme
 > > >  for the course of Quantum Mechanics which I like to read to PhD
 > > >   students next year.
 > > >   (Just for convenience I send already the text file of this programme
 > > > (see below the letter))
 > > >    The programme is written on the base of the template which
 > > >    Yurij Bazlov gave me
 > > > (this was the programme of the course of Tuomas Sahlsten)
 > > >
 > > >  You know that I had the first attempt to read
 > > >  these lectures the Autumn former year. This was unformal course.
 > > >    James Montaldi, Sasha Borovik and Yuri Bazlov
 > > >    attended this course.
 > > >
 > > >      Thank you.
 > > >
 > > >         Hovik
 > > >
 > > >
 > > >
 > > > %%%%%%%%%%%%%%%%%%%%%%%%%%%%%%%%%%%%%%%%%%%%%%%%%%%%%%%%%%%%%%%%%%%%%%%%%%%%
 > > >
 > > >               GRADUATE CURSE FOR PHD STUDENTS
 > > >
 > > >
 > > > ELEMENTS OF QUANTUM MECHANICS FOR MATHEMATICIANS
 > > >
 > > >
 > > >
 > > > This course counts as 33 hours of the taught component of a PGR
 > > > programme.
 > > >
 > > >     Total Time: 22 lectures over 11 weeks
 > > >  + up to 11 tutorial sessions.
 > > >
 > > >           Academic Year: 2018/2019
 > > >
 > > > Course leader: Hovhannes Khudaverdyan
 > > >
 > > >
 > > >
 > > > Unit co-ordinator: Hovhannes Khudaverdian
 > > >
 > > >
 > > >
 > > >
 > > >                       PURPOSE OF THE COURSE
 > > >
 > > >
 > > >
 > > > From  very beginning, the development of
 > > > Quantum Mechanics  had very strong interrelation with the
 > > > development of mathematics in XX century.
 > > > Nowdays the knowledge of quantum mechanics
 > > > is indispensable in many areas of mathematics.
 > > > This course is an attempt to deliver the main aspects of  Quantum
 > > > Mechanics
 > > > paying special attention
 > > > to   arising mathematical constructions. These include theory of spin and
 > > > angular momentum, represenations of the group $SO(3)$, uncertainty
 > > > principle,
 > > > Fourier transform, and generalized functions, quasi-classical
 > > > approximation,
 > > > and elements of quantum logic.
 > > >
 > > >
 > > >
 > > >
 > > >
 > > >                  POTENTIAL AUDIENCE
 > > >
 > > > Graduate students in pure applied mathematics and logic.
 > > >
 > > >
 > > >
 > > >                    PREREQUISITES
 > > >
 > > > A clear understanding of linear algebra and calculus is required. Some
 > > > knowledge of elements of functional analysis
 > > > and the Lagrangian and Hamiltonian formalism in classical mechanics
 > > > is desirable.
 > > >
 > > >
 > > >
 > > >
 > > >                STRUCTURE OF THE COURSE
 > > >
 > > >
 > > > Lectures - 22 hours
 > > >
 > > > Tutorials - 11 hours
 > > >
 > > >
 > > >
 > > >
 > > >                             READING LIST
 > > >
 > > > There are many excellent textbooks in Quantum Mechanics such as
 > > >
 > > >
 > > >  --- L.D. Landau, E.M. Lifshitz   {\it Quantum Mechanics:
 > > >      Non-Relativistic Theory (Volume 3)}
 > > >
 > > >
 > > >
 > > >   --- Leonard I. Schiff {\it Quantum Mechanics}
 > > >
 > > >
 > > >
 > > >    ---  Enrico Fermi {\it Notes on Quantum Mechanics}.
 > > >
 > > >
 > > >
 > > >
 > > >
 > > > Books for further reading and  special topics:
 > > >
 > > >    --- C.Piron {\it M\'echanique Quantique.
 > > >         Bases et applications}
 > > >
 > > >
 > > >     ---Leon A. Takhtajan:
 > > >   Stony Brook University, Stony Brook, NY, {\it Quantum Mechanics for
 > > > Mathematicians}
 > > >
 > > >
 > > >
 > > >
 > > >                        ASSESMENT
 > > >
 > > > 2 assignments
 > > >
 > > >
 > > >
 > > >
 > > >                  SYLLABUS
 > > >
 > > > --- Unitary space: complex linear space with Hermitian metric.
 > > > Self-adjoint operators in a unitary space. States and observables in
 > > > Quantum Mechanics. Measurement: commuting and non-commuting observables.
 > > > Cauchy-Bunyakovsky-Schwarz inequality and Heisenberg uncertainty
 > > > principle.
 > > >
 > > > ---  Wave-function in Quantum Mechanics and action in Classical
 > > > Mechanics. Schroedinger equation. Coordinate and momentum
 > > > representations. Harmonic oscillator.
 > > >
 > > >
 > > > --- Rotation and angular momentum. Spin of a particle.
 > > >     Irreducible representations of the group $SO(3)$
 > > >
 > > >
 > > > --- Perturbation theory: abrupt and adiabatic perturbations.
 > > > Adiabatic invariants in Quantum Mechanics and in Classical Mechanics.
 > > >
 > > >
 > > > --- Quasiclassical approximation in Quantum Mechanics and
 > > > Hamilton-Jacobi equation in Classical Mechanics. Fourier transform and
 > > > Legendre transform. The idea of Maslov index.
 > > >
 > > >
 > > > --- Elements of Quantum Logic. Modular lattice of questions
 > > > in Quantum Mechanics and distributive lattice of
 > > > questions in  Classical Mechanics.
 > > >
 > > > %%%%%%%%%%%%%%%%%%%%%%%%%%%%%%%%%%%%%%%%%%%%%%%%%%%%%%%%%%%%%%%%%%%%%%%%%%%
 > > >
 > > >
 > > >
 > > >
 > > >
 > > >
 > > >
 > > >                             Dr Hovhannes Khudaverdian
 > > >                         Senior Lecturer in Pure Mathematics
 > > >                              School of Mathematics
 > > >                           The University of Manchester
 > > >                       Oxford Road, Manchester  M13 9PL, UK
 > > >                         e-mail:khudian@manchester.ac.uk
 > > >                          tel 00(44)-(0)-161-200-36-82
 > > >
 > > >
 > >
 > > Best wishes, Mark
 > >
 > > ----------------------------------------------------------------------
 > >  /\/)ark /-(ambites
 > >
 > >  School of Mathematics        Email: Mark.Kambites@manchester.ac.uk
 > >  University of Manchester
 > >  Manchester M13 9PL             WWW: www.ma.man.ac.uk/~mkambites/
 > >  England
 > > ----------------------------------------------------------------------
 > >
 > 
 > 

Best wishes, Mark

----------------------------------------------------------------------
  /\/)ark /-(ambites

  School of Mathematics        Email: Mark.Kambites@manchester.ac.uk
  University of Manchester
  Manchester M13 9PL             WWW: www.ma.man.ac.uk/~mkambites/
  England
----------------------------------------------------------------------

