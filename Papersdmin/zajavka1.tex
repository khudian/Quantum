

\documentclass[12pt]{article}
\usepackage{amsmath,amsthm}


\usepackage{amsmath,amssymb,amsfonts,amsthm}

\begin{document}


\centerline 
{\bf Elements of Quantum Mechanics for mathematicians}

\smallskip

This course counts as 33 hours of the taught component 
of a PGR programme.

{\bf Total Time: 22 lectures over 11 weeks
 ( $+$ up to 11 tutorial sessions) Academic Year: 2018/2019

Course leader: Hovhannes Khudaverdyan

% Informal title: 


Unit co-ordinator: Hovhannes Khudaverdian  

%Teaching assistant  (if applicable):


}

\medskip

{\bf Purpose of the course}

 

From the very beginning the development of
Quantum Mechanics  had very strong interrelation with
developmnet of mathematics in XX century. 
 Nowdays the knowledge of qunatum mechanics
is indispensable in many areas of mathematics.
This course is an attempt to 
deliver the main aspects of  Quantum Mechanics,
paying the special attention
to mathematical constructions arising.



 
 Thedia, ty tut mog by nabrosatj dve tri frazy

**************************************************


****************************************************


\medskip
 
{\bf Potential audience}

Graduate students in pure applied mathematics and logic.


\medskip


{\bf  Prerequisites}

Mostly the clear understanding of linear algebra 
and calucluas is required. 
The knowledge of elements of Functional analysis
 and Lagrangian (Hamiltonian) formalism in classical mechanics
is desirable.

\bigskip

{\bf Structure of the course}


Lectures - 22 hours 

Tutorials - 11 hours 

\medskip

{\bf Reading list}

There are many excellent textbooks in Quantum Mechanics....
  

 --- L.D.Landau, E.M.Lifshitz   {\it Quantum Mechanics: 
     Non-Relativistic Theory (Volume 3)}


\smallskip

  --- Leonard I. Schiff {\it Quantum mechanics}

\smallskip

   ---  Enrico Fermi {\it Notes on Quantum Mechanics}

and many others


\medskip

  For more profound reading where you may find background
of special topics 

   --- C.Piron {\it M\'echanique quantique. 
        Bases et applications}

\medskip

    ---Leon A. Takhtajan: 
  Stony Brook University, Stony Brook, NY, 
{\it Quantum Mechanics for Mathematicians}

\bigskip



{\bf Assessment}

2 assignments

\medskip


{\bf Syllabus}

\begin{itemize}

 \item 
Unitary space: complex linear space with Hermitian metric.
Self-adjoint operators in a unitary space. 
States and observables in
Quantum Mechanics. Measurement: 
commuting and non-commuting observables.
Cauchy-Bunyakovsky-Schwarz inequality and Heisenberg uncertainty principle.
 
\item  Wave-function in Quantum Mechanics and action in Classical
Mechanics. Schroedinger equation. Coordinate and momentum
representations. Harmonic oscillator.
 	

\item Rotation and angular momentum. Spin of a particle.
    Irreducibnle representations of group $SO(3)$


\item 
Perturbation theory: abrupt and adiabatic perturbations.
Adiabatic invariants in Quantum Mechanics and in Classical Mechanics.
 

\item Quasiclassical approximation in Quantum Mechanics and
Hamilton-Jacobi equation in Classical Mechanics. 
Fourier transform and
Legendre transform. Maslov index.
 	

\item Elements of Quantum Logic. Modular lattice of questions
in Quantum Mechanics and distributive lattice of
questions in  Classical Mechanics.

\end{itemize}
%%%%%%%%%%%%%%%%%%%%%%%%%%%%%%%%%%%%%%%%%%%%%%%%%%%%%%%%%%%

\end{document}


%%%%%%%%%%%%%%%%%%%%%%%%%%%%%%%%%%%%%%%%%%%%%%%%%%%%%%%%%%%%%%%%%
                   Topics in Quantum Mechanics
                   H.M.Khudaverdian
   Preliminary programme

 	1. Unitary space: complex linear space with Hermitian metric.
Self-adjoint operators in a unitary space. States and observables in
Quantum Mechanics. Measurement: commuting and non-commuting observables.
Cauchy-Bunyakovsky-Schwarz inequality and Heisenberg uncertainty principle.
 	2. Wave-function in Quantum Mechanics and action in Classical
Mechanics. Schroedinger equation. Coordinate and momentum
representations. Harmonic oscillator.
 	3. Rotation and angular momentum. Spin of a particle.
 	4. Perturbation theory: abrupt and adiabatic perturbations.
Adiabatic invariants in Quantum Mechanics and in Classical Mechanics.
 	5. Quasiclassical approximation in Quantum Mechanics and
Hamilton-Jacobi equation in Classical Mechanics. Fourier transform and
Legendre transform. Maslov index.
 	6. Elements of Quantum Logic. Modular lattice of questions
in Quantum Mechanics and distributive lattice of
questions in  Classical Mechanics.

%%%%%%%%%%%%%%%%%%%%%%%%%%%%%%%%%%%%%%%%%%%%%%%%%%%%%%%%%%%%%%%%%%



