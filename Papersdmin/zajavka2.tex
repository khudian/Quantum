\documentclass[12pt]{article}
\usepackage{amsmath,amsthm}


\usepackage{amsmath,amssymb,amsfonts,amsthm}

\begin{document}

\centerline {\bf  GRADUATE COURSE FOR PHD STUDENTS}

\bigskip

\centerline 
{\bf Elements of Quantum Mechanics for mathematicians}

\smallskip

This course counts as 33 hours of the taught component 
of a PGR programme.

{\bf Total Time: 22 lectures over 11 weeks
 ( $+$ up to 11 tutorial sessions) Academic Year: 2018/2019

Course leader: Hovhannes Khudaverdyan

% Informal title: 


Unit co-ordinator: Hovhannes Khudaverdian  

%Teaching assistant  (if applicable):


}

\medskip

{\bf Purpose of the course}

 

From  very beginning, the development of
Quantum Mechanics  had very strong interrelation with the
development of mathematics in XX century. 
 Nowdays the knowledge of quantum mechanics
is indispensable in many areas of mathematics.
This course is an attempt to 
deliver the main aspects of  Quantum Mechanics
paying special attention
to   arising mathematical constructions. 
These include theory of spin and angular momentum,  
represenations of the group $SO(3)$, 
uncertainty principle, Fourier transform, 
and generalized functions, 
quasi-classical approximation, and elements of quantum logic.   

 
\medskip
 
{\bf Potential audience}

Graduate students in pure applied mathematics and logic.


\medskip


{\bf  Prerequisites}

A clear understanding of linear algebra 
and calculus is required. 
Some knowledge of elements of functional analysis
 and the Lagrangian and Hamiltonian formalism 
in classical mechanics
is desirable.

\bigskip

{\bf Structure of the course}


Lectures - 22 hours 

Tutorials - 11 hours 

\medskip

{\bf Reading list}

There are many excellent textbooks in Quantum Mechanics such as
  

 --- L.D. Landau, E.M. Lifshitz   {\it Quantum Mechanics: 
     Non-Relativistic Theory (Volume 3)}


\smallskip

  --- Leonard I. Schiff {\it Quantum Mechanics}

\smallskip

   ---  Enrico Fermi {\it Notes on Quantum Mechanics}.



\medskip

Books for further reading and  special topics: 

   --- C.Piron {\it M\'echanique Quantique. 
        Bases et applications}

\medskip

    ---Leon A. Takhtajan: 
  Stony Brook University, Stony Brook, NY, 
{\it Quantum Mechanics for Mathematicians}

\bigskip



{\bf Assessment}

2 assignments

\medskip


{\bf Syllabus}

\begin{itemize}

 \item 
Unitary space: complex linear space with Hermitian metric.
Self-adjoint operators in a unitary space. 
States and observables in
Quantum Mechanics. Measurement: 
commuting and non-commuting observables.
Cauchy-Bunyakovsky-Schwarz inequality and Heisenberg uncertainty principle.
 
\item  Wave-function in Quantum Mechanics and action in Classical
Mechanics. Schroedinger equation. Coordinate and momentum
representations. Harmonic oscillator.
 	

\item Rotation and angular momentum. Spin of a particle.
    Irreducible representations of the group $SO(3)$


\item 
Perturbation theory: abrupt and adiabatic perturbations.
Adiabatic invariants in Quantum Mechanics and in Classical Mechanics.
 

\item Quasiclassical approximation in Quantum Mechanics and
Hamilton-Jacobi equation in Classical Mechanics. 
Fourier transform and
Legendre transform. The idea of Maslov index.
 	

\item Elements of Quantum Logic. Modular lattice of questions
in Quantum Mechanics and distributive lattice of
questions in  Classical Mechanics.

\end{itemize}

\end{document}

